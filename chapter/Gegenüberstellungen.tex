% !TeX root = ../Thesis.tex

\subsection*{Harmonisierung vs. Abstraktion (Niklas Hardes)}

Harmonisierung und Abstraktion sind zwei verschiedene Konzepte in der Software-Entwicklung.

Harmonisierung bezieht sich auf die Anpassung von Daten oder Prozessen, um sie konsistent und einheitlich zu machen bzw. an Standards anzupassen. Das Ziel der Harmonisierung ist es, Inkonsistenzen oder Unstimmigkeiten in Daten oder Prozessen zu beseitigen, um sicherzustellen, dass sie miteinander kompatibel und leicht zu verarbeiten sind.

Abstraktion hingegen bezieht sich auf die Verringerung der Komplexität von etwas, indem man nur die wesentlichen Eigenschaften behält und die unbedeutenden Details ignoriert. Das Ziel der Abstraktion ist es, ein Problem oder eine Aufgabe in kleinere, einfachere Teile zu zerlegen, um es leichter zu verstehen und zu lösen.

Kurz gesagt, Harmonisierung sorgt dafür, dass Daten und Prozesse miteinander kompatibel sind, während Abstraktion dafür sorgt, dass Probleme und Aufgaben leichter verstanden und gelöst werden können.