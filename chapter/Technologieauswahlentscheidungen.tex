%!TEX root = ../Thesis.tex

\section{Technologieentscheidungen}

\subsection*{Nicolas Groß}

Bei der Entwicklung des Frontend haben wir uns für das Ionic-Framework entschieden. Das Ionic-Framework ist ein Open-Source UI Toolkit für die Entwicklung von Hybrid-Apps. Hybrid-Apps sind Anwendungen, welche auf Web-Technologien wie HTML, CSS und JavaScript basieren und die nativen APIs und Funktionen des Betriebssystems nutzen können.\footcite[.vgl]{HybrideWebApp}  Ionic bietet Vorteile wie Performance-Optimierung, Cross-Plattform und die automatische Bereitstellung einer Hellen und Dunklen Ansicht.Ionic bietet zahlreiche anpassbare UI-Komponenten, welche helfen die Strukturierung und Gestaltung der Webanwendung vorzunehmen. Zudem bietet Ionic eine ausführliche Dokumentation und eine große Community, wodurch das Einbauen unbekannter Komponenten schnell und einfach ist. Ionic unterstützt neben dem klassischen JavaScript die Frameworks Angular, React und Vue.\footcite[.vgl]{Ionic2013}

Hierbei fiel unsere Entscheidung aus erfahrungsgründen auf Angular. Angular ist ein von Google entwickeltes auf TypeScript basierendes Open-Source Webframework zur Entwicklung von Mobile und Desktop Webanwendungen. Angular stellt ein Command-Line-Interface bereit, welches Entwicklern bei der Erstellung und Entwicklung der Projekte unterstützt. Die Entwicklung von Single-Page-Apps wird durch Angular Routing ermöglicht, wodurch Webanwendungen schneller auf Benutzereingaben reagieren und die Ladezeiten zwischen einzelnen Seiten stark vermindern. Zudem können Inhalte der Webseite dynamisch geladen und angepasst werden. Angular bietet genau wie Ionic eine ausführliche Dokumentation, Lehrunterlagen und eine aktive Community. Durch Angular ergibt sich die Möglichkeit umfangreiche Bibliotheken einzubinden und deren Ressourcen für die Entwicklung des Projektes zu nutzen.\footcite[.vgl]{Angular2016}


\subsection{Docker (Robert Hesselmann)}

Bei der Planung unseres Softwaresystems haben wir uns für Docker als Laufzeitumgebung entschieden. Dadurch kann unsere Software auf allen System laufen die Docker unterstützen. Der Ablauf vom Push eines Commits bis zum Deploy der Software, geht die Software du die Schritte des Bau