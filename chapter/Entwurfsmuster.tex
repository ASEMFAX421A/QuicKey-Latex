% !TeX root = ../Thesis.tex

\section{Entwurfsmuster}

\subsection*{Nicolas Groß}

Bei Anwendungen wie in unserem Fall, einer Single-Page-Application (SPA), werden alle Teile auf einmal geladen, um dem Nutzer ein möglichst flüssiges Erlebnis zu bieten. Dadurch werden häufig auch unbenutzte Module der Anwendung geladen. Bei kleineren Projekten stellt dies keine Probleme da, allerdings kann es bei größeren Projekten zu langen Ladezeiten führen. Um dem entgegenzuwirken, bietet sich Lazy Loading. Beim Lazy Loading werden Inhalte einer Anwendung erst geladen, sobald sie benötigt werden. Dadurch werden Wartezeiten beim Start der Anwendung verkürzt.
In Angular haben Module die Möglichkeit via Lazy Loading geladen zu werden. Durch den entsprechenden Code im Routing-Module wird das Lazy Loading auf die Module angewendet.

\begin{figure}[bht]
\begin{lstlisting}[caption=Codeausschnitt des Tabs Routingmodule,language=json]
{
  path: 'home',
  loadChildren: () => import('../tab1/tab1.module')
                        .then(m => Tab1PageModule),
  data: {
    title: 'Home', 
    icon: 'home'
  }
}
\end{lstlisting}
%\footnoterule{}
%\footnotesize{Casts have been omitted for the sake of readability}
\end{figure}

Wie in diesem Codeausschnitt zu sehen, wird das Modul erst beim Aufruf der Route importiert und geladen. Somit wurde Lazy Loading in Angular-Routing implementiert.\footcites[.vgl]{LazyLoading2021}

Neben dem Routing bedienen wir uns noch zwei weiter Male den Lazy Loading. Der von Ionic bereitgestellte HTML-Tag „<ion-img>“, welche zum Anzeigen Bildern verwendet wird, lädt Bilder standardmäßig über Lazy Loading. Den deutlichsten unterschied der Ladezeiten weist allerdings die InfinityScroll der „Search“-Seite auf. Bei einer Suche werden die Einträge paketweise der Liste hinzugefügt. Weiter Einträge werden erst der Liste erst hinzugeladen, sobald der Nutzer bis kurz vorm unteren Ende der Liste scrollte.\footcites[.vgl]{Ionic2013b}

\subsection*{Fluent-Interfaces (Niklas Hardes)}

In der Softwareentwicklung sind Fluent-Interfaces ein Entwurfsmuster, das darauf abzielt, den Code lesbarer und verständlicher zu machen, indem es die Verkettung von Methoden ermöglicht. Sie sind besonders nützlich für die Erstellung von komplexen und hierarchischen Objekten.

In unserem Projekt haben wir für das Backend die Sprache C\# gewählt. In dieser Sprache werden diese Ketten dadurch ermöglicht, dass Methoden das eigene Objekt zurückgeben, wodurch weitere Methoden direkt hintereinander aufgerufen werden können.

Wir haben dieses Muster unter anderem benutzt, um HTTP Anfragen auf einfache Art konstruieren zu können.
Dazu haben wir die Klasse CustomHttpRequest angelegt, welche dann Methoden wie die SetHeader besitzt.

\begin{figure}[bht]
\begin{lstlisting}[caption=Codeausschnitt von CustomHttpRequest (1),language=csh]
public CustomHttpRequest SetHeader(string key, string value)
{
    this._headers ??= new Dictionary<string, string>();
    this._headers[key] = value;
    return this;
}
\end{lstlisting}
%\footnoterule{}
%\footnotesize{Casts have been omitted for the sake of readability}
\end{figure}

Diese Methode konfiguriert bzw. modifiziert die Header des Requests und gibt sich selber zurück, wodurch weitere Methoden aufgerufen werden können. Ein vollständiger Aufruf könnte dann wie folgt aussehen.

\begin{figure}[bht]
\begin{lstlisting}[caption=Codeausschnitt von CustomHttpRequest (2),language=csh]
_requester.BuildRequest(HttpMethod.Get, "https://api.steampowered.com/ISteamApps/GetAppList/v2/")
            .SetHeader("User-Agent", "QuicKey Worker")
            .SetCookie("testCookie", "testValue")
            .Request()
\end{lstlisting}
%\footnoterule{}
%\footnotesize{Casts have been omitted for the sake of readability}
\end{figure}

In dem Konstruktor bzw. der Factory Methode werden die benötigten Attribute wie die Methode und die Uri, welche zwingend benötigt werden direkt übergeben. Diese Methode gibt dann ein Objekt der Klasse CustomHttpRequest zurück. Bei diesem Objekt wird dann mithilfe eines Fluent-Interfaces ein exemplarischer Header und Cookie gesetzt. Am Ende wird dann die Request Methode aufgerufen, wodurch die Kette abgeschlossen, die Anfrage gestellt und das Ergebnis der Anfrage zurückgegeben wird. Dies ist dann kein CustomHttpRequest mehr, sondern eine HttpResponseMessage.

\subsection*{Stellvertreter (Proxy) (Robert Hesselmann)}

Wir benutzen den Stellvertreter (Proxy) in unserem Backend um Anfragen an die jeweiligen Keystores zu senden. Unser Proxy verwendet mehrere IP-Adressen, die zum tatsächlichen Ausführen der Anfrage benutzt wird. Somit können viele Anfragen gleichzeitig an die Keystores gesendet werden. Alle Daten fließen gebündelt wieder an das Backend. Danach kann das Backend mit der Verarbeitung der Seiten beginnen.