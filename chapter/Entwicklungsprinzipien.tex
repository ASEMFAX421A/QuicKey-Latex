%!TeX root = ../Thesis.tex

\section{Entwicklungsprinzipien}

\subsection*{Nicolas Groß}

Modularisierung ist die Unterteilung eines Systems in kleinere Module. Die Modularisierung bietet Eigenschaften, welche bei der Entwicklung und Wartung von Software, Vorteile einbringen.  Zu diesen Vorteilen zählen Verständlichkeit, Kombinierbarkeit, Lokalität und die parallele Entwicklung. Die Verständlichkeit einer Software wird durch die Modularisierung verbessert, da einzelne Bestandteile weitgehend unabhängig von anderen Bestandteilen gekapselt und verständlich sind. Die einzelnen Module einer Software können auf unterschiedliche Arten miteinander kombiniert werden, um neue Systeme zusammenzufügen. Dies bietet den Vorteil das Module idealerweise unabhängig vom restlichen System funktionieren und wiederverwendet werden können. Durch die Lokalität zeichnet sich aus, dass Änderungen an einzelnen Modulen keine größeren Änderungen im Gesamtsystem zufolge haben. Ein entscheidender Vorteil für die Entwicklung im Team ist die Möglichkeit zur parallelen Entwicklung. Hierbei können einzelne Teammitglieder an unterschiedlichen Modulen arbeiten, ohne mit den Entwicklungen der anderen Mitglieder zu kollidieren. Das System wird dann zu einem späteren Zeitpunkt zusammengesetzt.\footcite[.vgl]{Schmidauer2002}

Durch Angular ist eine hohe Modularisierung bereits von Beginn des Projekts gegeben, da die Angular-CLI Komponenten und Services in einzelnen Dateien erzeugen. Auch die Projektstruktur wird von der Angular-CLI angepasst, um Komponenten und Services anzulegen. Die einzelnen Komponenten können durch die Verwendung von Variablen wiederverwendbar gestaltet werden, um diese beliebig zu kombinieren. Die Änderungen an einer Komponente haben hierbei in der Regel keinen Einfluss auf andere Komponenten oder Services, wodurch Änderungen meist leicht umgesetzt werden können und dennoch Projektweit geltend sind.

\subsection*{Single-Responsibility Prinzip (Niklas Hardes)}

Bei dem Single-Responsibility Prinzip lautet die Kern-Definition \glqq Ein Modul sollte einem, und nur einem, Akteur gegenüber verantwortlich sein.\grqq{}. Es ist ein wichtiger Bestandteil der SOLID-Prinzipien der objektorientierten Programmierung. Es bedeutet, dass jede Klasse, Methode oder Funktion in einem Programm nur eine einzige Verantwortung hat.

Dem entgegen steht ein sogenanntes God-Object. Es hat viele Verantwortungen und wirkt sich auf unterschiedliche Bereiche aus.
Das macht es schwierig, dieses zu verstehen, zu testen und zu warten. Des Weiteren kann es aufwendig sein, alle Auswirkungen im Blick zu behalten, die eine Änderung an diesem mit sich bringen würde. Insofern ist es sehr zu empfehlen, die Verantwortlichkeiten gering zu halten bzw. im Idealfall nur eine einzige zu haben.

Ebenfalls gibt es Auswirkungen auf Abhängigkeiten zwischen Klassen. Wenn eine Klasse mehrere Verantwortungen hat, kann es viele Abhängigkeiten zu anderen Klassen geben, die die Wartbarkeit des Codes beeinträchtigen können. Eine Klasse mit einer einzigen Verantwortung hat jedoch in der Regel weniger Abhängigkeiten, was die Wartbarkeit des Codes verbessert.

Zusammenfassend, unter Einhaltung des SRP sollte jede Klasse, Methode oder Funktion nur eine einzige Verantwortung haben. Dies erleichtert es, Änderungen an einer Klasse vorzunehmen, ohne Auswirkungen auf andere Teile des Systems zu haben, und die Abhängigkeiten zwischen Klassen zu minimieren.

Das beste Beispiel liefert hierzu unsere Implementation des ThreadedWebSocket im Namespace QuicKey.Business.Web. Es besitzt eine geringe Anzahl an Abhängigkeiten, welche ebenfalls über Interfaces ausgetauscht werden können und somit gut testbar sind. Seine einzige Verantwortlichkeit liegt dabei, die Kommunikation über ein WebSocket zu ordnen, zu steuern und anderen Klassen zur Verfügung zu stellen.